\svnid{$Id: ds_guide.tex 1 2015-08-30 15:39:28Z rodriqu_dd $}

\chapter{Installatie\label{chap:install}}

\section{Inleiding}
Deze sectie beschrijft de installatie procedure van de applicatie Wettelijk Toetsing Instrumentarium (WTI). Voor de installatie van WTI is in ieder geval het installatie bestand (WTI Setup*.msi) nodig. Daarnaast worden er enkele eisen aan het besturingssysteem gesteld (zie \Autoref{sec:sysrequirements}). Na installatie (paragraaf 3) kan WTI worden opgestart via het startmenu (onder de map Deltares) of door te dubbelklikken op het WTI icoontje (\Fref{fig:fig2.1}) op het bureaublad. 


\begin{figure} [H]
	\centering
		\includegraphics{figures/chapter_installation/wti_desktop_icon}
	\caption{Wettelijk Toetsing Instrumentarium (WTI) icoon.}
	\label{fig:fig2.1}
\end{figure}


\section{Systeemeisen}
\label{sec:sysrequirements}
Voor een goed functioneren van WTI is het wenselijk (of in sommige gevallen nodig) om een computer te hebben die minimaal voldoet aan de volgende eisen:
\begin{itemize}
	\item Microsoft Windows 7 hoger
	\item Microsoft .NET Framework versie 4.0 of hoger
	\item Minimaal een Intel Pentium III/800 MHz processor (of vergelijkbaar)
	\item Minimaal 256(???) MB RAM (1 GB RAM aanbevolen)
	\item Minimale beeldscherm resolutie van 1024x768 pixels
\end{itemize}




\section{WTI installeren en opstarten}
De procedure om de applicatie Wettelijk Toetsing Instrumentarium (WTI) te installeren wordt uitgelegd. Ook hoe de applicatie hersteld of verwijderd worden kan.

